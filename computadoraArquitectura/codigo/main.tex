\documentclass[conference]{IEEEtran}
\usepackage{graphicx}
\usepackage{newtxtext,newtxmath}
\usepackage{amsmath}
\usepackage[letterpaper, margin=2.54cm]{geometry}
\usepackage[spanish]{babel}
\usepackage{fancyhdr}
\usepackage[table]{xcolor}
\usepackage{array}
\usepackage[style=ieee]{biblatex}
\usepackage{float}
\addbibresource{srcs.bib}
\pagestyle{fancy}
\fancyhead[R]{\thepage}

\begin{document}

\title{Proyecto 1.\\Computadora de Desarrollo Mobile.}

\author{
    \IEEEauthorblockN{Angel Navas, Jose Natera}
    \IEEEauthorblockA{\textit{Departamento de Computación, Universidad de Carabobo}}
    \IEEEauthorblockA{Materia: Arquitectura del Computador \\
    Profesor: Jose Canache}
}

\maketitle

\begin{abstract}

Se propone el diseño de una computadora especializada para el desarrollo de aplicaciones móviles. Se prioriza la selección de \textit{hardware} que ofrezca un equilibrio entre desempeño, eficiencia y costo, con el objetivo de garantizar una arquitectura capaz de soportar cargas de trabajo exigentes propias del desarrollo \textit{mobile} aprovechando al máximo las cotas del presupuesto.

\end{abstract}

\begin{IEEEkeywords}
Computer Architecture; Mobile Development; Hardware Components; Workstation Build; Performance; Cost Efficiency.
\end{IEEEkeywords}

\section{Introducción}

Este proyecto surge con el objetivo de diseñar una computadora optimizada para un nicho de uso específico, aprovechando al máximo un presupuesto total de 3000\$. Tras analizar distintas áreas de aplicación, se optó por enfocar el diseño en el desarrollo de aplicaciones móviles. En consecuencia, la mayor parte del presupuesto se destinó a componentes que ofrecieran una alta capacidad de memoria y un elevado rendimiento de procesamiento. Esta decisión se fundamenta en las exigencias propias de este ámbito, donde el uso intensivo de entornos de desarrollo, emuladores y herramientas de depuración demanda un \textit{hardware} capaz de sostener sesiones de elevadas de trabajo de manera eficiente y estable en el tiempo.

\section{Hardware}

\subsection{Procesador - Intel Core i9-14900K}

Los 24 núcleos permiten la ejecución simultánea de IDEs pesados, múltiples emuladores, navegadores y procesos de compilación en segundo plano sin degradar la experiencia del usuario. Su capacidad de alcanzar los 6GHz garantiza que las tareas de un solo hilo, comunes en la ejecución de scripts y herramientas de desarrollo, se completen con la menor latencia posible.
\begin{center}
    \includegraphics[height=2cm]{imagenes1/intel.jpg}
\end{center}

\subsection{Placa Madre - ROG Strix Z790-E Gaming WiFi 6E}

Esta placa utiliza el chipset Z790, diseñado específicamente para extraer el máximo potencial de los procesadores Intel de 14va generación. Ofrece WiFi 6E para transferencias de datos ultra rápidas y un robusto sistema de fases de poder que garantiza que el i9 reciba energía estable sin sobre-calentar la placa. Además de ser nativa con DDR5, permitiendo aprovechar el manejo de grandes volúmenes de datos en RAM.
\begin{center}
    \includegraphics[height=2cm]{imagenes1/asus.png}
\end{center}

\subsection{Memoria RAM - Crucial DDR5 5600MT/s 96GB ($24\text{GB}\times4$)}
Android Studio y los emuladores son extremadamente dependientes de la memoria. Con 96GB, un desarrollador puede mantener abiertos simultáneamente el IDE, varios contenedores Docker, navegadores y herramientas de diseño sin que el sistema sufra retardos y al operar a 5600MT/s, se alinea perfectamente con la velocidad de memoria base del procesador.
\begin{center}
    \includegraphics[height=2cm]{imagenes1/crucial.png}
\end{center}

\subsection{Tarjeta de Video - MSI GeForce RTX 3060 Ventus 12GB}
Los 12GB de memoria de video son fundamentales para la aceleración por hardware de los emuladores de Android, permitiendo que las interfaces de las aplicaciones se rendericen con fluidez.
\begin{center}
    \includegraphics[height=2cm]{imagenes1/3060.png}
\end{center}

\subsection{Almacenamiento - Samsung 990 Pro 2TB NVMe M.2 PCIe 4.0}
El indexado de archivos en los IDEs y el despliegue de máquinas virtuales dependen de la velocidad del disco. Este modelo garantiza que la constante escritura de logs y archivos temporales de compilación no degrade el disco prematuramente.
\begin{center}
    \includegraphics[height=2cm]{imagenes1/ssd.jpg}
\end{center}

\subsection{Fuente de Poder - Apevia 1000W 80 Plus Gold}
Provee un colchón de energía suficiente para cubrir los picos de consumo del i9 y la GPU, operando en su punto de máxima eficiencia para reducir el calor generado. La certificación Gold asegura una entrega de energía limpia, crucial para proteger componentes de gama alta ante las fluctuaciones eléctricas comunes.
\begin{center}
    \includegraphics[height=2cm]{imagenes1/psu.jpg}
\end{center}

\subsection{Refrigeración y Pasta Térmica - Thermaltake 360 y Pasta Térmica LK-10}
Un Thermaltake de 360 nos ayudara a evitar el thermal throttling en el i9, permitiendo que el procesador mantenga sus frecuencias altas por más tiempo.
La pasta térmica LK-10 optimiza la transferencia de calor entre el procesador y el bloque de enfriamiento, reduciendo la temperatura de operación en varios grados críticos. Todo esto nos garantiza unas temperaturas agradables en su uso en un medio plazo.

\begin{center}
    \includegraphics[height=2cm]{imagenes1/360.jpg}
\end{center}

\subsection{Gabinete - Fantech Aero CG83}
Ofrece el espacio necesario para instalar la placa ATX y el Sistema de Enfriamiento, asegurando que el flujo de aire sea suficiente para mantener los componentes internos en rangos de temperatura.

\begin{center}
    \includegraphics[height=2cm]{imagenes1/case.jpg}
\end{center}
\newpage
\section{Presupuesto}

Con un presupuesto máximo de $3000$\$ los gastos en componentes serian los siguientes (En caso de ser varias unidades, el precio mostrado se calcula multiplicando el precio individual por el numero de unidades, por ejemplo la \textit{RAM} son dos unidades de dos módulos cada una):

{\rowcolors{2}{gray!70}{lightgray}
\begin{center}
\begin{tabular}{m{2.4cm}|m{2.3cm}|m{2.1cm}}
     Componente & Nombre & Precio(\$) \\
     \hline
     \textbf{Memoria RAM} & Crucial DDR5 5600 96gb (2*24*2)\cite{crucial_ddr5} & $2\times350\$=700\$$ \\
     \textbf{Placa Base} &  Asus Rog Strix Z790-e Gaming Wifi 6e\cite{asus_z790} & $550\$$ \\
     \textbf{Procesador} &  Intel Core I9 14900k\cite{intel_i9_14900k} & $530\$$ \\
     \textbf{Tarjeta de Video} &  MSI GeForce RTX 3060 Ventus\cite{msi_rtx3060} & $420\$$ \\
     \textbf{Almacenamiento} & SSD Samsung 990 Pro 2TB M.2 2280 Pcie 4.0 Nvme\cite{samsung_990pro} & $260\$$ \\
     \textbf{Fuente de Poder} & Fuente De Poder 1000w Apevia Certificada 80 Plus Gold Pc Atx\cite{apevia_1000w} & $207.13\$$ \\
     \textbf{Refrigeración} & Enfriamiento Liquido Thermaltake 360 RGB\cite{thermaltake_360} & $200\$$ \\
     \textbf{Gabinete} & Case Fantech Aero CG83 FRGB\cite{fantech_case} & $127.60\$$ \\
     \textbf{Pasta Térmica} & LK-10 10w/mk\cite{lk-10} & $5\$$ \\
     \hline \rowcolor{white}
     & \textbf{Total($\Sigma$): } & $2999.73$
\end{tabular}
\end{center}}
\appendix
\subsection{Registro visual de productos consultados}

\begin{figure}[h]
\centering
\includegraphics[width=0.7\linewidth]{imagenes2/cpu.png}
\caption{Procesador Intel Core i9-14900K}
\end{figure}

\begin{figure}[h]
\centering
\includegraphics[width=0.7\linewidth]{imagenes2/3060.png}
\caption{Tarjeta Gráfica MSI GeForce RTX 3060 Ventus 2X 12G OC}
\end{figure}

\begin{figure}[h]
\centering
\includegraphics[width=0.7\linewidth]{imagenes2/ram.png}
\caption{Memoria RAM DDR5 Crucial de 5600MT/s}
\end{figure}

\begin{figure}[h]
\centering
\includegraphics[width=0.7\linewidth]{imagenes2/mother.png}
\caption{Tarjeta Madre ASUS ROG Strix Z790-E Gaming WiFi 6E}
\end{figure}

\begin{figure}[h]
\centering
\includegraphics[width=0.7\linewidth]{imagenes2/psu.png}
\caption{Fuente de Poder Apevia 1000W Certificada 80 Plus Gold}
\end{figure}

\begin{figure}[h]
\centering
\includegraphics[width=0.7\linewidth]{imagenes2/ssd.png}
\caption{SSD Samsung 990 Pro 2TB M.2 NVMe PCIe 4.0}
\end{figure}

\begin{figure}[h]
\centering
\includegraphics[width=0.7\linewidth]{imagenes2/case.png}
\caption{Gabinete Fantech Aero CG83 FRGB}
\end{figure}

\begin{figure}[h]
\centering
\includegraphics[width=0.7\linewidth]{imagenes2/360.png}
\caption{Refrigeración Liquida Thermaltake 360}
\end{figure}

\begin{figure}[h]
\centering
\includegraphics[width=0.7\linewidth]{imagenes2/pasta.png}
\caption{Pasta Termina LK-10 10w/mk}
\end{figure}
\onecolumn
\printbibliography


\end{document}
